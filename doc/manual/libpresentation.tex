\chapter{The Old Presentation Library}

\begin{python}
pyscript.lib.presentation
\end{python}

In addition to the newer \vrb{pyscript.lib.present} library, there also
exists the old \vrb{pyscript.lib.presentation} library.  This is not so
object-oriented in usage, but it works nevertheless, and hasn't been
completely deprecated.  The \vrb{presentation} library can be used to create
posters and talks, which can then be used to ``wow'' your colleauges at your
next conference.

\section{Common Objects}

These are useful objects for both posters and talks.

\subsection{TeXBox()}

Typeset some \LaTeX{} within a fixed width box.

\subsection{Box\_1()}

A box of fixed width.  Items added to it are aligned vertically and
separated by a specified padding.

\section{Creating a talk or seminar}

The Talk() object.

\subsection{set\_title()}

Or you can just set the title attribute directly.

\subsection{set\_authors()}
Or you can just set the title attribute directly.

\subsection{set\_speaker()}

\subsection{set\_address()}


The Slide() object.

\subsection{set\_title()}

\begin{python}
slide5.set_title(r"Moooo!)"
\end{python}

You can also set the title by explicitly assigning the title attribute:
\begin{python}
slide5.title = r"Moooo!"
\end{python}

Not much of a difference between the two I know, but some people like to
call a set\_ function and others like to set the attribute directly, so
we're catering to both kinds of people.

\subsection{add\_heading()}

\subsection{add\_fig()}

Add a diagram to the slide (this is handled differently to add\_epsf() eh?)

\subsection{add\_epsf()}

Add an encapsulated Postscript file to the slide.  Works very similarly to
add\_fig().

\subsection{Styles for talks and seminars}

You can create your own custom talk slide style and call this at the
beginning when you create the Talk() object.  The style can then be put in
your .pyscript directory, and called whenever you make a slide.  New styles
can be added and so a wider range of talk backgrounds and styles can be
made.  That was a shit explanation eh?

\section{Creating a poster}

