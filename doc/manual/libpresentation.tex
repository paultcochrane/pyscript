% $Id$

\chapter{The Old Presentation Library}

\begin{python}
pyscript.lib.presentation
\end{python}

In addition to the newer \vrb{pyscript.lib.present} library, there also
exists the old \vrb{pyscript.lib.presentation} library.  This is not so
object-oriented in usage, but it works nevertheless, and hasn't been
completely deprecated.  The \vrb{presentation} library can be used to create
posters and talks, which can then be used to ``wow'' your colleauges at your
next conference.

\section{Common Objects}

These are useful objects for both posters and talks.

\subsection{TeXBox()}

Typeset some \LaTeX{} within a fixed width box.

\subsection{Box\_1()}

A box of fixed width.  Items added to it are aligned vertically and
separated by a specified padding.

\section{Creating a talk or seminar}

\subsection{The Talk() object}

The first thing you will need to do when you start writing a seminar
presentation is to instantiate the \vrb{Talk()} object.  This object defines
some overall parameters, attributes and styles for the talk as a whole.
After you set these parameters for your particular talk, then you only need
to worry about adding \vrb{Slide()} objects.  To set up the \vrb{Talk()}
object for your talk you merely need to do this:
\begin{python}
talk = Talk()
\end{python}
More interesting things happen when we add styles to the talk, but more on
that later.  If you want to know about that now, go to \Sec{sec:styles}.

The next thing you probably want to do is to give your talk a title.  There
are two ways to do this: with the \vrb{set\_title()} method, or by merely
setting the \vrb{title} attribute of the instantiated \vrb{Talk()} object.
In other words you can either do this:
\begin{python}
talk.set_title(r"This is my talk")
\end{python}
or this:
\begin{python}
talk.title = r"This is my talk"
\end{python}

It is common that there are many people who have contributed to a particular
piece of work being discussed in the seminar or talk, and consequently there
will be more than one ``author'' of the talk.  However, there is usually
only one person presenting the talk, and so we have two separate attributes
for these situations, namely the \vrb{authors} and \vrb{speaker} attributes.
To set the name of the authors contributing to the talk, either use the
\vrb{set\_authors()} method, or set the attribute directly, like so:
\begin{python}
talk.set_authors(r"Tom, Dick, and Harry")
\end{python}
or:
\begin{python}
talk.authors = r"Tom, Dick and Harry"
\end{python}
For the speaker, this is almost an identical procedure, just use either the
\vrb{set\_speaker()} method, or set the \vrb{speaker} attribute directly.

You are likely to be representing a business or institute of some form, so
it is best to give their address.  To do provide this information to the
\vrb{presentation} library so that it can place the text appropriately, just
use the \vrb{set\_address()} method or set the \vrb{address} attribute of
the instantiated \vrb{Talk()} object.

That's basically it as far as the \obj{Talk()} object itself goes.  The main
amount of work is in producing the individual slides of the presentation,
which is what we discuss next.

\subsection{The Slide() object}

The \vrb{Slide()} object defines a particular slide; one creates a new
\vrb{Slide()} object for each slide in the presentation, calling them all at
the end in the \vrb{render()} function to generate the entire talk.  

The first slide in your talk will be the titlepage.  To generate the
titlepage do something like this:
\begin{python}
titlepage = Slide(talk)  # instantiate a Slide object of the current 'talk'
titlepage.titlepage = True  # tell the library that this is the titlepage
\end{python}
There exists a convenience function to tell the library that this is the
titlepage, and, you guessed it, it's called \vrb{set\_titlepage()}.

Slides usually have a title, some sequence of headings, possibly a figure
(defined in \pyscript for example), or an imported EPS image, and possibly
some equations.  The \vrb{presentation} library provides methods for doing
all these things.  To add a slide to the presentation, one must instatiate a
new \vrb{Slide} object, passing to it the current \vrb{Talk} object, like
so:
\begin{python}
intro = Slide(talk)
\end{python}

To add a title to the slide~\footnote{Note that this is
\textbf{not} the title of the talk!}, one can use the \vrb{set\_title()}
method (this time of the \vrb{Slide} class), or set the \vrb{title}
attribute directly:
\begin{python}
intro.set_title(r"Introduction")
\end{python}
or:
\begin{python}
intro.title = r"Introduction"
\end{python}
There isn't much of a difference between the two I know, but some people
like to call a \vrb{set\_} function and others like to set the attribute
directly, so we're catering to both kinds of people.

To add other things like headings, figures and epsf images to your slide,
you just need to use one of the relevant \vrb{add\_*()} functions.  In other
words, to add a heading use the \vrb{add\_heading()} method.  This method
takes two arguments, the first argument is the level of the heading (there
are currently three separately defined levels of headings in the library)
and the second argument is the heading to add.  For instance:
\begin{python}
intro.add_heading(1, r"What are we talking about?")
intro.add_heading(2, r"Some stuff")
intro.add_heading(2, r"Some other stuff")
intro.add_heading(3, r"Something more specific to some other stuff")
\end{python}
Adding a heading puts one of the predefined bullets in front of the heading,
typesets the text at a predefined size and indentation dependent upon the
heading level.  You can change these settings by defining your own style, or
by setting one of the myriad attributes of the talk object itself, for more
information see \Sec{sec:styles}.

If you want to place a diagram generated from \pyscript code to your slide,
and have it automatically positioned by the library, then use the
\vrb{add\_fig()} method.  For instance, if you've produced earlier in your
\pyscript script a diagram called \vrb{mydiag} then to add it to the slide,
merely use:
\begin{python}
intro.add_fig(mydiag)
\end{python}
You can set the location of the figure by passing one of the \pyscript anchor
locations as an optional argument.  For example,
\begin{python}
intro.add_fig(mydiag, ne=intro.area.ne - P(1,1))
\end{python}
will locate the diagram in the ``north-east'' corner of the page one
centimetre from the right-hand edge, and one centimetre from the top edge.
You can set the width of the diagram as well by specifying the \vrb{width}
option:
\begin{python}
intro.add_fig(mydiag, width=12, c=intro.area.c)
\end{python}
which will make the diagram 12cm wide, and centre it on the current page.

Similarly, one can add diagrams or images that already exist in EPS files.
To do this use the \vrb{add\_epsf()} method like so:
\begin{python}
intro.add_epsf(file="myEpsFile.eps")
\end{python}
The \vrb{add\_epsf()} method processes the anchor location and width
options in the same way the \vrb{add\_fig()} method does.

At the end of your script you'll want to render the output of your slides,
and generate the talk, to do this use the \vrb{render()} function of
\pyscript in the usual manner:
\begin{python}
# render it!
render(
    titlepage,
    intro,
    another_slide,
    file="mytalk.ps")
\end{python}
This will generate a Postscript document called \vrb{mytalk.ps} in the same
directory as that in which the \pyscript script was run.  Note that this is
a Postscript file and not an EPS file; this means that the output has
multiple pages as one would hope for a seminar!

To actually give your seminar there are several tools you can use.  One of
the most common is to turn the Postscript into PDF via \texttt{epstopdf} or
some similar tool, and then use the full screen mode of Adobe Acrobat
Reader~\cite{acroread} to display the talk.  Alternatively, you might like to
use a program like \texttt{pspresent}~\cite{pspresent}.

\subsection{Styles for talks and seminars}
\label{sec:styles}

\section{Creating a poster}

