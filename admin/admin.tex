% $Id$

\documentclass[12pt,a4paper]{article}

\begin{document}

\section{Epydoc}

Epydoc is used to build the API documentation of pyscript.  It strips out
relevant information from the python code and the docstrings in the python
code to generate a set of web pages documenting the different classes and
functions defined.

The command used to run epydoc is (from the pyscript base directory):
\begin{verbatim}
epydoc --html -o api -n pyscript pyscript
\end{verbatim}

\section{Pylint}

Pylint is a handy tool to check the quality of the code.  It is like the old
lint program for C, and gives a lot of checking and helpful output.  To
generate html output from pylint use the following command from the pyscript
main directory:
\begin{verbatim}
pylint --enable-similarities=n --ignore=fonts --html=y pyscript > pylint.pyscript.html
\end{verbatim}

\section{Release procedure}

\begin{enumerate}
\item Update CHANGES file
\item Update MANIFEST.in
\item Update BUGS file
\item Update README file
\item Update TODO file
\item Make sure user manual is up to date
\item Make sure epydoc isn't throwing any nasty errors, and if so, fix them
\item Make sure pylint isn't picking up errors, if so, fix them (warnings
and refactoring messages are probably ok...)
\item Verify that build installs correctly (use e.g.~debootstrap, rpmstrap)
\item Tag sources to current version (format: e.g. v0-6-0; command: cvs tag
vx-y-z, where x, y and z are the relevant version digits)
\item Build distribution (\texttt{python setup.py sdist})
\item Do a file release on sf.net
\item Test that the release downloads and builds correctly
\item Make a .rpm (Fedora Core 3 and Fedora Core 4) of the release and 
add to file release on sf.net (ftp to upload.sf.net; follow release
instructions)
\item Make a .deb (Sarge) of the release and add to file release on sf.net
\item Make a Gentoo ebuild and post to Gentoo Bugzilla
\item Make a News announcement on sf.net pyscript project page
\item Update website to report and reflect current version
\item Update user manual on website
\item Build the API from the distribution, \textbf{not} from the repository 
version.  This is because the repository version has stuff in it that isn't
yet public, and the testing stuff for instance doesn't need to be part of
what is put up on the web
\item Update API on website (use epydoc; see above)
\item Post an announcement of release to comp.lang.python
\item Update/add entry at the Python Cheeseshop:
\texttt{http://cheeseshop.python.org/pypi}
\item Update \texttt{freshmeat.net} entry
\end{enumerate}

\end{document}
