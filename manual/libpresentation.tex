% Copyright (C) 2002-2006  Alexei Gilchrist and Paul Cochrane
% 
% This program is free software; you can redistribute it and/or
% modify it under the terms of the GNU General Public License
% as published by the Free Software Foundation; either version 2
% of the License, or (at your option) any later version.
%
% This program is distributed in the hope that it will be useful,
% but WITHOUT ANY WARRANTY; without even the implied warranty of
% MERCHANTABILITY or FITNESS FOR A PARTICULAR PURPOSE.  See the
% GNU General Public License for more details.
%
% You should have received a copy of the GNU General Public License
% along with this program; if not, write to the Free Software
% Foundation, Inc., 59 Temple Place - Suite 330, Boston, MA  02111-1307, USA.

% $Id$

\chapter{The Old Presentation Library}

\begin{python}
pyscript.lib.presentation
\end{python}

In addition to the newer \vrb{pyscript.lib.present} library, there also
exists the old \vrb{pyscript.lib.presentation} library.  This is not quite
so object-oriented in usage, but it works nevertheless, and hasn't been
completely deprecated.  The \vrb{presentation} library can be used to create
posters and talks, which can then be used to ``wow'' your colleauges at your
next conference.

\section{Common Objects}

These are useful objects for both posters and talks.

\subsection{TeXBox()}

Typeset some \LaTeX{} within a fixed width box.

\subsection{Box\_1()}

A box of fixed width.  Items added to it are aligned vertically and
separated by a specified padding.

\section{Creating a talk or seminar}

\subsection{The Talk() object}

The first thing you will need to do when you start writing a seminar
presentation is to instantiate the \vrb{Talk()} object.  This object defines
some overall parameters, attributes and styles for the talk as a whole.
After you set these parameters for your particular talk, then you only need
to worry about adding \vrb{Slide()} objects.  To set up the \vrb{Talk()}
object for your talk you merely need to do this:
\begin{python}
talk = Talk()
\end{python}
More interesting things happen when we add styles to the talk, but more on
that later.  If you want to know about that now, go to \Sec{sec:styles}.

The next thing you probably want to do is to give your talk a title.  There
are two ways to do this: with the \vrb{set\_title()} method, or by merely
setting the \vrb{title} attribute of the instantiated \vrb{Talk()} object.
In other words you can either do this:
\begin{python}
talk.set_title(r"This is my talk")
\end{python}
or this:
\begin{python}
talk.title = r"This is my talk"
\end{python}

It is common that there are many people who have contributed to a particular
piece of work being discussed in the seminar or talk, and consequently there
will be more than one ``author'' of the talk.  However, there is usually
only one person presenting the talk, and so we have two separate attributes
for these situations, namely the \vrb{authors} and \vrb{speaker} attributes.
To set the name of the authors contributing to the talk, either use the
\vrb{set\_authors()} method, or set the attribute directly, like so:
\begin{python}
talk.set_authors(r"Tom, Dick, and Harry")
\end{python}
or:
\begin{python}
talk.authors = r"Tom, Dick and Harry"
\end{python}
For the speaker, this is almost an identical procedure, just use either the
\vrb{set\_speaker()} method, or set the \vrb{speaker} attribute directly.

You are likely to be representing a business or institute of some form, so
it is best to give their address.  To do provide this information to the
\vrb{presentation} library so that it can place the text appropriately, just
use the \vrb{set\_address()} method or set the \vrb{address} attribute of
the instantiated \vrb{Talk()} object.

It is possible that your business or institute has a logo that you'd like to
use.  If so, convert it to an EPS file, and you can add it to the talk using
the \vrb{add\_logo()} method like so:
\begin{python}
talk.add_logo("myepslogo.eps", height=2)
\end{python}
You can set the height of the logo using the \vrb{height} attribute as shown
in the example above.

That's basically it as far as the \obj{Talk()} object itself goes.  The main
amount of work is in producing the individual slides of the presentation,
which is what we discuss next.

\subsection{The Slide() object}

The \vrb{Slide()} object defines a particular slide; one creates a new
\vrb{Slide()} object for each slide in the presentation, calling them all at
the end in the \vrb{render()} function to generate the entire talk.  

The first slide in your talk will be the titlepage.  However, since getting
you to make a new \vrb{Slide()} object just to generate the title page is
silly (well, you've just given the title page all it needs to know eh?) the
library automatically generates the title page slide for you.

Slides usually have a title, some sequence of headings, possibly a figure
(defined in \pyscript for example), or an imported EPS image, and possibly
some equations.  The \vrb{presentation} library provides methods for doing
all these things.  To add a slide to the presentation, one must instatiate a
new \vrb{Slide} object, passing to it the current \vrb{Talk} object, like
so:
\begin{python}
intro = Slide(talk)
\end{python}

To add a title to the slide~\footnote{Note that this is
\textbf{not} the title of the talk!}, one can use the \vrb{set\_title()}
method (this time of the \vrb{Slide} class), or set the \vrb{title}
attribute directly:
\begin{python}
intro.set_title(r"Introduction")
\end{python}
or:
\begin{python}
intro.title = r"Introduction"
\end{python}
There isn't much of a difference between the two I know, but some people
like to call a \vrb{set\_} function and others like to set the attribute
directly, so we're catering to both kinds of people.

To add other things like headings, figures and epsf images to your slide,
you just need to use one of the relevant \vrb{add\_*()} functions.  In other
words, to add a heading use the \vrb{add\_heading()} method.  This method
takes two arguments, the first argument is the level of the heading (there
are currently three separately defined levels of headings in the library)
and the second argument is the heading to add.  For instance:
\begin{python}
intro.add_heading(1, r"What are we talking about?")
intro.add_heading(2, r"Some stuff")
intro.add_heading(2, r"Some other stuff")
intro.add_heading(3, r"Something more specific to some other stuff")
\end{python}
Adding a heading puts one of the predefined bullets in front of the heading,
typesets the text at a predefined size and indentation dependent upon the
heading level.  You can change these settings by defining your own style, or
by setting one of the myriad attributes of the talk object itself, for more
information see \Sec{sec:styles}.

If you want to place a diagram generated from \pyscript code to your slide,
and have it automatically positioned by the library, then use the
\vrb{add\_fig()} method.  For instance, if you've produced earlier in your
\pyscript script a diagram called \vrb{mydiag} then to add it to the slide,
merely use:
\begin{python}
intro.add_fig(mydiag)
\end{python}
You can set the location of the figure by passing one of the \pyscript anchor
locations as an optional argument.  For example,
\begin{python}
intro.add_fig(mydiag, ne=intro.area.ne - P(1,1))
\end{python}
will locate the diagram in the ``north-east'' corner of the page one
centimetre from the right-hand edge, and one centimetre from the top edge.
You can set the width of the diagram as well by specifying the \vrb{width}
option:
\begin{python}
intro.add_fig(mydiag, width=12, c=intro.area.c)
\end{python}
which will make the diagram 12cm wide, and centre it on the current page.

Similarly, one can add diagrams or images that already exist in EPS files.
To do this use the \vrb{add\_epsf()} method like so:
\begin{python}
intro.add_epsf(file="myEpsFile.eps", c=intro.area.c, width=14)
\end{python}
The \vrb{add\_epsf()} method processes the anchor location, width and height
options in the same way the \vrb{add\_fig()} method does.

You might like to add some text to your slide and position it arbitrarily on
the page, as opposed to have \vrb{presentation.py} work out where to put it
for you (as is done with the \vrb{add\_heading()} method).  Therefore, there
is the convenience function \vrb{add\_text()}.  Here is an example:
\begin{python}
intro.add_text("Hello there!", e=intro.area.e-P(-2,-2), scale=2)
\end{python}
Just to be different, the \vrb{add\_text()} method has a scale attribute as
opposed to a height or width attribute.  This is really silly, and should
probably be changed.  If you read this sentence, please put a feature
request to change this on the \href{http://pyscript.sf.net}{pyscript} web
page :-).

At the end of your script you'll want to make the talk in its entirety, to
do this use the \vrb{make()} method of the \vrb{Talk()} class like so:
\begin{python}
# make it!
talk.make(
    intro,
    another_slide,
    file="mytalk.ps")
\end{python}
This will generate a Postscript document called \vrb{mytalk.ps} in the same
directory as that in which the \pyscript script was run.  Note that this is
a Postscript file and not an EPS file; this means that the output has
multiple pages as one would hope for a seminar!

To actually give your seminar there are several tools you can use.  One of
the most common is to turn the Postscript into PDF via \ttt{ps2pdf} or some
similar tool, and then use the full screen mode of Adobe Acrobat
Reader~\cite{acroread} to display the talk.  Alternatively, you might like
to use a program like \ttt{pspresent}~\cite{pspresent}.  Please note that
due to a bug in Ghostscript version 7 (and possibly below) if there is
insufficient text on a page, the postscript will be converted by
\ttt{ps2pdf} incorrectly to give a portait-orientated page, as opposed to a
landscape page.  The postscript will view correctly in Ghostview (and so
will the pdf incidentally), however, the pdf will view incorrectly in
\ttt{xpdf} and \ttt{acroread}.  This bug seems to have been fixed in
Ghostscript version 8.

\subsection{Styles for talks and seminars}
\label{sec:styles}

To change your talk style from the default you can specify one of the
predefined styles by passing the \vrb{style} option to the \vrb{Talk()}
class on instantiation.  For instance:
\begin{python}
talk = Talk(style="prosper-darkblue")
\end{python}
which will load the style that looks a lot like the ``darkblue'' style of
the \href{prosper}{prosper} \LaTeX{} package.  To load the style, \pyscript
will look in either the \ttt{~/.pyscript/styles/} directory or the current
directory for a python file whose file name will be the name of the style
with `\ttt{.py}' appended.

If you are feeling really keen, you can write your own style.  One of the
best ways to do this is to copy and then modify one of the ones provided
with the \pyscript distribution.  Let's go through the details of the
``prosper-darkblue'' style now.

\begin{python}
# talk style for PyScript, following the Darkblue design of prosper

HOME = os.path.expandvars("$HOME")
stylesDir = HOME + "/.pyscript/styles/"

# set the foreground and background colour of the title text of the talk
self.title_fg = Color('white')
self.title_bg = Color('white')

# set the talk title's text style
self.title_textstyle = r"\bf\sf"

# set the text style for the text of who is giving the talk
self.speaker_textstyle = r"\sf"

# set the colour and text style of the address of the speaker of the talk
self.address_fg = Color('white')
self.address_textstyle = r"\sf"

# set the colour and text style of the authors of the talk (not necessarily
# the speaker of the talk)
self.authors_fg = Color('white')
self.authors_textstyle = r"\sf"

# set the colour and text style of the title of the *slide*
self.slide_title_fg = Color('lightgray')
self.slide_title_textstyle = r"\bf"

# set the colour, scale, textstyle, bullet and indent type for a level 1 heading
self.headings_fgs[1] = Color('white')
self.headings_scales[1] = 3
self.headings_textstyle[1] = r"\sf"
self.headings_bullets[1] = Epsf(file=stylesDir+"redbullet.eps").scale(0.2,0.2)
self.headings_indent[1] = 0

# set the colour, scale, textstyle, bullet and indent type for a level 2 heading
self.headings_fgs[2] = Color('white')
self.headings_scales[1] = 2.5
self.headings_textstyle[2] = r"\sf"
self.headings_bullets[2] = Epsf(file=stylesDir+"greenbullet.eps").scale(0.15,0.15)
self.headings_indent[2] = 0.5

# set the colour, scale, textstyle, bullet and indent type for a level 3 heading
self.headings_fgs[3] = Color('white')
self.headings_scales[1] = 2.2
self.headings_textstyle[3] = r"\sf"
self.headings_bullets[3] = Epsf(file=stylesDir+"yellowbullet.eps").scale(0.1,0.1)
self.headings_indent[3] = 1

# set the colour, textstyle and scale for placed text
self.text_scale = 3.0
self.text_fg = Color('white')
self.text_textstyle = r"\sf"
\end{python}
%stopzone  % help for vim

First off we work out where the styles directory is, and since it should be
in \ttt{~/.pyscript/styles} we define this with the \ttt{stylesDir}
variable.  Note that this is a python file like any other, all it's doing is
expecting you to add a certain set of values that \pyscript's
\vrb{presentation.py} library knows about.  

Next we set the title foreground and background colour by setting the
\vrb{title\_fg} and \vrb{title\_bg} variables.  The text style is bold
(\ttt{\\bf}) and a sans serif (\ttt{\\sf}) font.  Note that this uses
\LaTeX{} to specify these styles as internally \pyscript \vrb{TeX} objects
are used.  If this is a problem for you, just send us a bug report or
feature request on the \href{http://pyscript.sf.net}{pyscript} web page and
we'll try and fix it for you as soon as we can.

Next the text style for the speaker element of the title page is set, and
the colour and text style for the authors and address elements.

Now we process the elements for the slides themselves.  We need to set a
foreground colour and text style for the title of the slide.  We also need
to specify the scale, foreground colour, text style bullet and indentation
amount to use for the three levels of headings defined in the
\vrb{presentation.py} library.  Note that one has to specify for which level
the style is being set by the square brackets: e.g.~\vrb{headings\_fg[1]}.  

One doesn't have to set the bullets to a text or \LaTeX kind of string, but
also to EPS files, which is what we have done in the above example where
red, green and yellow circles are used for the various heading levels.
We've scaled them here (using the \pyscript \vrb{scale} method) so that they
have different sizes and one can more easily tell when viewing the talk that
one level has precedence over the other.

If one wishes to place arbitrary text on the page, one can specify the
default scale, colour and text style here too.

Styles can be found in the \vrb{contrib/} directory of the \pyscript
distribution.

\section{Creating a poster}

\subsection{The Poster() object}

The very first thing to do when making a new poster is to instantiate the
\vrb{Poster()} object.  You do this like so:
\begin{python}
poster = Poster(size="a4")
\end{python}
Note that the size of the poster has been set here explicitly.  A handy
thing about using postscript is that if you want an a0 poster, you just need
to change the size paramter to ``a0'' and the size of the poster will
change, but not the actual amount of postcript.  The reason we add this size
declaration to the poster class is because it is often handy to have a4 size
versions of your poster when you are at a poster session at a conference to
give to people who have viewed your a0 size poster.

Your poster will then require a title, a list of authors, an address of the
insititution you are representing and the abstract that you submitted (or
are using) for the poster.  You set these properties using the relevant
\vrb{set\_()} methods of the \vrb{Poster()} class.  For instance:
\begin{python}
poster.set_title("My poster")

poster.set_authors("Me, him, and her")

poster.set_address("Over the hills, and far away")

poster.set_abstract("""
I should have written something better when I applied for the conference: I
might have got a talk instead...
""")
\end{python}

One often wants to put logos onto a poster.  The \vrb{Poster()} class lets
you do this easily by using the \vrb{add\_logo()} method.  Just supply the
name of an EPS file and the height you want to use for the logo, and the
class will place it at the top of your poster for you.  E.g.:
\begin{python}
poster.add_logo("my_first_logo.eps", height=1.2)
poster.add_logo("my_second_logo.eps", height=1.2)
\end{python}
The first logo is placed at the top left-hand corner.  The next logo will be
placed in the top right-hand corner.  If you specify more logos then they
will be distributed evenly across the top of the poster.  If you want, you
can also specify a list of logo names all at the one time, instead of having
to call \vrb{add\_logo()} several times.  You do this with the
\vrb{add\_logos()} method:
\begin{python}
poster.add_logos("my_first_logo.eps", "my_second_logo.eps", height=1.2)
\end{python}
Obviously, when you use this method, all logos are given the same height.

A poster is usually split up into columns.  For a portrait-orientated
poster, one usually has two columns; for a landscape-orientated poster three
columns.  A column is made up of one or more ``column boxes'' which define
the actual content in the poster.  So, to make your first column, you need
to make a new \vrb{Column()} object:
\begin{python}
col1 = Column(poster)
\end{python}
Note that we have passed the current \vrb{poster} object through as an
argument.  This is so that any styles defined in the poster can flow through
to the subobjects.  We will eventually add this column to the poster as a
whole, but not yet, as we need to fill the column with some content first.
To do this instantiate a \vrb{ColumnBox()} object like so:
\begin{python}
intro = ColumnBox(poster)
intro.set_title("Introduction")
intro.add_TeXBox(r"""
Here is some \LaTeX{}
""")
\end{python}
Again, the \vrb{poster} object is passed in as an argument so we can make
use of previously defined styles.  We then set the title of the column box
with the \vrb{set\_title()} method, and then added a \vrb{TeXBox()} object,
which can contain arbitrary \LaTeX{} expressions, with the
\vrb{add\_TeXBox()} method.

There are more things than just \vrb{TeXBox}es that you can add to a column
box; you can add an already-defined \pyscript object, or an EPS file.  Say
you want to align two images up together, and display them side by side
within the column box.  To do this you would set up an \vrb{Align()} object,
append the EPS files as \vrb{Epsf()} objects and then add the \vrb{Align()}
object to the column box with the \vrb{add\_object()} method.  E.g.:
\begin{python}
# the figures
fig1 = Epsf("fig1.eps")
fig2 = Epsf("fig2.eps")

# make the Align object
figs = Align(a1="ne", a2="nw", angle=90, space=0.2)
figs.append(fig1, fig2)

# add the Align object to the column box
intro.add_object(figs)
\end{python}

The \vrb{add\_epsf()} method is merely a convenience function so that you
don't have to define an \vrb{Epsf()} object first and then add it with the
\vrb{add\_object()} method.  You can do all that work if you want to though!

We are now ready to add the column box to the column.  We do this with the
\vrb{add\_box()} method of the \vrb{Column()} object:
\begin{python}
col1.add_box(intro)
\end{python}

We can now add the column to the poster itself, and yes, you guessed it, to
do this we use the \vrb{add\_column()} method of the \vrb{Poster()} object:
\begin{python}
poster.add_column(col1)
\end{python}
Since this is the first column, it will automatically be the left-most
column.  Any other columns that you add will be added to the right of it.
So, for instance, if one were making a poster in portrait mode then one
would simply add two columns; the \vrb{Poster()} class should handle
aligning most of the bits and pieces for you.

For more complex examples, have a look in the \vrb{examples/} directory of
the \pyscript distribution.  The \vrb{Poster()} class is a complete rewrite
and extension of the older \vrb{Poster\_1()} class, so it is likely that
there will be problems with the way that it has been designed and so the
interface might change slightly in the future.  If you are having problems,
feel free to email the developers!  If you read this sentence, you might
want to email the developers anyway!  :-)

\subsection{Styles for posters}
\label{sec:poster-styles}

To change your poster style from the default you can specify one of the
predefined styles by passing the \vrb{style} option to the \vrb{Poster()}
class on instantiation.  For instance:
\begin{python}
poster = Poster(size="a4", style="ccp2004-poster")
\end{python}
which will load the style that PTC used at the Conference on Computational
Physics in 2004 (hence ccp2004).  Just like with styles for talks, to load
the style, \pyscript will look in either the \ttt{~/.pyscript/styles/}
directory or the current directory for a python file whose file name will be
the name of the style with `\ttt{.py}' appended.

If you are feeling really keen, you can write your own style.  One of the
best ways to do this is to copy and then modify one of the ones provided
with the \pyscript distribution.

Styles can be found in the \vrb{contrib/} directory of the \pyscript
distribution.
