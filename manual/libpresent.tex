% Copyright (C) 2002-2006  Alexei Gilchrist and Paul Cochrane
% 
% This program is free software; you can redistribute it and/or
% modify it under the terms of the GNU General Public License
% as published by the Free Software Foundation; either version 2
% of the License, or (at your option) any later version.
%
% This program is distributed in the hope that it will be useful,
% but WITHOUT ANY WARRANTY; without even the implied warranty of
% MERCHANTABILITY or FITNESS FOR A PARTICULAR PURPOSE.  See the
% GNU General Public License for more details.
%
% You should have received a copy of the GNU General Public License
% along with this program; if not, write to the Free Software
% Foundation, Inc., 59 Temple Place - Suite 330, Boston, MA  02111-1307, USA.

% $Id$

\chapter{Presentation Library}

\begin{python}
pyscript.lib.present
\end{python}

It's straightforward to create whole posters in \pyscript. 
Talks can also be simply created. 

\section{Common Objects}

These are useful objects for both posters and talks.

%--------------------------------------------------------------------------
\subsection{Box()}
\begin{python}
class Box(Group, Rectangle):
    pad = .2

    width = None
    height = None
\end{python}

This places a box around an object. \Verb|pad| gives the amount of padding
around the object. \Verb|width| and \Verb|height| can be set also and these
will override the calculated values. 

%--------------------------------------------------------------------------
\subsection{TeXArea()}
\begin{python}
class  TeXArea(Group):
    width = 9.4
    iscale = 1
    align = "w"
\end{python}
    
This will typeset some \LaTeX{} within a fixed width minipage environment.
the width of the minipage is set with the \Verb|width| attribute which must
be supplied. The initial scale of the text can be set with \Verb|iscale| as
per the \Verb|TeX| object. The \Verb|align| attribute is used when the
\LaTeX{} doesn't fill the minipage --- an invisible rectangle is added with
the minipage aligned according to this attribute.

%--------------------------------------------------------------------------
\section{Posters}

\begin{python}
class Poster(Page, VAlign):
    size = "A0"
    orientation = "portrait"

    bg = Color('DarkSlateBlue')

    space = 1

    topspace = 2
	
    def background(self)
\end{python}


%--------------------------------------------------------------------------
\section{Talks}


\begin{python}
class Pause(object)
class Talk(Pages):
    def append(self, *slides_raw):
class EmptySlide(Page):

    title = None
    orientation = "Landscape"
    size = "screen"

    def flatten(self, thegroup=None, objects=[]):
    def append(self, *items, **options):
    def append_n(self, *items):
    def append_s(self, *items):
    def append_e(self, *items):
    def append_w(self, *items):
    def append_c(self, *items):

    def make_back(self):
    def make_title(self):

	def clear(self):
    def make(self, page=1, total=1):
\end{python}

